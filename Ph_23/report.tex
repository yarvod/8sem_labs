\documentclass[a4paper]{article}
\usepackage[warn]{mathtext}
\usepackage[utf8]{inputenc}
\usepackage[T2A]{fontenc}
\usepackage[english,russian]{babel}
\usepackage{booktabs}
\usepackage{multicol}
\usepackage{fancyhdr}
\usepackage{graphicx}
\usepackage{microtype}
\usepackage{wrapfig}
\usepackage{amsmath}
\usepackage{floatflt}
\usepackage{geometry} \geometry{verbose,a4paper,tmargin=2cm,bmargin=2cm,lmargin=1.5cm,rmargin=1.5cm}
\usepackage{float}
\usepackage{amssymb}
\usepackage{caption}
\usepackage{epsfig}
\usepackage{newunicodechar}

\begin{document}

\graphicspath{ {pictures/} }
\begin{center}
    {\scshape\Large Лабораторная работа по квантовой электронике} \par

    \

    {\huge\bfseries № 23} \par 

    \

    {\large Яромир Водзяновский Б04-855а}
\end{center}

\

\
\textbf{Цель работы:} \par 
\begin{enumerate}
    \item Ознакомление с принципами применения магнитооптических методов для исследования прозрачных магнетиков. 
    \item Определение магнитных параметров исследуемого образца (коэрцетивная сила, поле насыщения)
    \item Изучение оснорв и принципов применения вычислительной техники для организации  автоматизированного сбора и анализа данных физического эксперимента
\end{enumerate}



\end{document}
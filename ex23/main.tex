\documentclass[a4paper]{article}
\usepackage[warn]{mathtext}
\usepackage[utf8]{inputenc}
\usepackage[T2A]{fontenc}
\usepackage[english,russian]{babel}
\usepackage{booktabs}
\usepackage{multicol}
\usepackage{fancyhdr}
\usepackage{graphicx}
\usepackage{microtype}
\usepackage{wrapfig}
\usepackage{amsmath}
\usepackage{floatflt}
\usepackage{geometry} \geometry{verbose,a4paper,tmargin=2cm,bmargin=2cm,lmargin=1.5cm,rmargin=1.5cm}
\usepackage{float}
\usepackage{amssymb}
\usepackage{caption}
\usepackage{epsfig}
\usepackage{newunicodechar}
\usepackage{color}

\begin{document}

\graphicspath{ {pictures/} }
\begin{center}
    {\scshape\Large Приборы полпроводниковой микро- и наноэлектроники} \par

    \

    {\huge\bfseries Эказменационное билет №23.} \par 

    \

    {\large Яромир Водзяновский Б04-855а}
\end{center}

\

\textbf{Доказать теорему о запрете клонирование неизветсного квантового бита. К каким последствиям приводит эта теорема для квантового компьютера и квантовой коммункации?}


Теорема о запрете клонирования — утверждение квантовой теории о невозможности создания идеальной копии произвольного неизвестного квантового состояния. 
Теорема была сформулирована Вуттерсом, Зуреком и Диэксом в 1982 году и имела огромное значение в области квантовых вычислений, квантовой теории информации и смежных областях.

Состояние одной квантовой системы может быть запутанным с состоянием другой системы. Например, создать запутанное состояние двух кубитов можно с помощью 
однокубитного преобразования Адамара и двухкубитного квантового вентиля C-NOT. Результатом такой операции не будет клонирование, поскольку результирующее 
состояние нельзя описать на языке состояний подсистем (состояние является нефакторизуемым). Клонирование — это такая операция, в результате которой создается 
состояние, являющееся тензорным произведением идентичных состояний подсистем.



\textit{Доказательство}

Пусть мы хотим создать копию системы $A$, которая находится в состоянии с. Возьмем систему $B$ с тем же самым гильбертовым пространством, 
находящуюся в начальном состоянии $|e \rangle_B$. Начальное состояние не должно зависеть от состояния $|\psi\rangle_A$, тк оно нам не известно. 
Составная система опишется тензорным произведением состояний подстистем:

\begin{equation}
    | \psi \rangle_A \otimes |e \rangle_B \equiv  | \psi \rangle_A  |e \rangle_B
\end{equation}

С составной системой можно произвести два различных действия.

\begin{enumerate}
    \item Измерить ее состояние, что приведет к необратимому переходу системы в одно из собственных состояний 
        измеряемой наблюдаемой и к частичной потере информации об исходном состоянии системы $A$.
    \item Применение унитарного преобразования $U$, должным образом настроим гамильтониан системы. 
        Оператор $U$ будет клонировать состояние ситемы, если:
        \begin{itemize}
            \item $U | \psi \rangle_A  |e \rangle_B = | \psi \rangle_A  |\psi \rangle_B$
            \item и $U | \phi \rangle_A  |e \rangle_B = | \phi \rangle_A  |\phi \rangle_B$
            \item $\forall \; |\phi \rangle \; \&  \; |\psi \rangle$
        \end{itemize}
\end{enumerate}

Согласно определению унитарного оператора $U$ сохраняется скалярное произведение:

\begin{equation}
    \langle e |_B \langle \phi |_A U^{\dagger} U |  \psi \rangle_A | e \rangle_B = \langle \phi |_B \langle \phi |_A |  \psi \rangle_A | \psi \rangle_B
\end{equation}

то есть

\begin{equation}
    \langle \phi | \psi \rangle = \langle \phi | \psi \rangle^2
\end{equation}

Из этого следует, что либо $| \phi \rangle = | \psi \rangle$, либо состояния $|\phi \rangle$ и $|\psi \rangle$ ортогональны 
Таким образом, операция $U$ не может клонировать произвольное квантовое состояние.

Теорема о запрете клонирования доказана.


\end{document}
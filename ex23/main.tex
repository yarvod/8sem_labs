\documentclass[a4paper]{article}
\usepackage[warn]{mathtext}
\usepackage[utf8]{inputenc}
\usepackage[T2A]{fontenc}
\usepackage[english,russian]{babel}
\usepackage{booktabs}
\usepackage{multicol}
\usepackage{fancyhdr}
\usepackage{graphicx}
\usepackage{microtype}
\usepackage{wrapfig}
\usepackage{amsmath}
\usepackage{floatflt}
\usepackage{geometry} \geometry{verbose,a4paper,tmargin=2cm,bmargin=2cm,lmargin=1.5cm,rmargin=1.5cm}
\usepackage{float}
\usepackage{amssymb}
\usepackage{caption}
\usepackage{epsfig}
\usepackage{newunicodechar}
\usepackage{color}

\begin{document}

\graphicspath{ {pictures/} }
\begin{center}
    {\scshape\Large Приборы полпроводниковой микро- и наноэлектроники} \par

    \

    {\huge\bfseries Эказменационное билет №23.} \par 

    \

    {\large Яромир Водзяновский Б04-855а}
\end{center}

\

\textbf{Доказать теорему о запрете клонирование неизветсного квантового бита. К каким последствиям приводит эта теорема для квантового компьютера и квантовой коммункации?}
\par 




\end{document}
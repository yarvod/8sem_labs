\documentclass[14pt,a4paper]{article}
\usepackage[warn]{mathtext}
\usepackage[utf8]{inputenc}
\usepackage[T2A]{fontenc}

\usepackage[english,russian]{babel}
\usepackage{multicol}
\usepackage{fancyhdr}
\usepackage{graphicx}
\usepackage{microtype}
\usepackage{wrapfig}
\usepackage{amsmath}
\usepackage{floatflt}
\usepackage{pdfpages}
\usepackage{geometry} \geometry{verbose,a4paper,tmargin=2cm,bmargin=2cm,lmargin=1.5cm,rmargin=1.5cm}
\usepackage{float}
\usepackage{amssymb}
\usepackage{caption}
\usepackage{epsfig}
\usepackage{newunicodechar}
\newcommand{\angstrom}{\textup{\AA}}

\usepackage{indentfirst}
\usepackage{misccorr}
\usepackage{subcaption}
\captionsetup{compatibility=false}
\usepackage{wrapfig}
\usepackage{amsmath}
\usepackage{float}
\usepackage{amssymb}
\usepackage{color}
\usepackage{lscape}
\usepackage{hvfloat}
\usepackage{amsfonts}
\usepackage{euscript}
\usepackage{textcomp}
\usepackage{mathtext}
\usepackage{latexsym}
\usepackage{xcolor}
\usepackage{hyperref}
\usepackage{booktabs}
\usepackage[version =3]{mhchem}
\usepackage{commath}
\usepackage{gensymb}
\usepackage{fancyhdr}
\usepackage[normalem]{ulem}
\newcommand{\RomanNumeralCaps}[1]
    {\MakeUppercase{\romannumeral #1}}

\definecolor{linkcolor}{HTML}{000BFF} % цвет ссылок
\definecolor{urlcolor}{HTML}{000BFF} % цвет гиперссылок

\begin{document}
	\begin{titlepage}
		\begin{center}
		МИНИСТЕРСТВО ОБРАЗОВАНИЯ И НАУКИ РОССИЙСКОЙ ФЕДЕРАЦИИ\\
		\vspace{0.5cm}
		\footnotesize{\Large{МОСКОВСКИЙ ФИЗИКО-ТЕХНИЧЕСКИЙ ИНСТИТУТ}}\\
		\vspace{0.15cm}
		\footnotesize{\Large{национальный исследовательский университет}}\\
		\vspace{6cm}
		{\LARGE
		\textbf{Задачи по курсу}}\\
		\large{\textit{квантовая оптика}}\\
		\vspace{8cm}
		\begin{flushright}
			Выполнил\\
			студент 855 группы\\
			ФФКЭ МФТИ\\
			Атепалихин Артемий Алексеевич
		\end{flushright}
		\vfill
		Долгопрудный\\
		\the\year\:
		\end{center}
	\end{titlepage}
	
	\pagestyle{fancy} 
    \fancyhead[L]{Атепалихин АА}
    \fancyhead[C]{квантовая оптика}
    \fancyhead[R]{\textit{задачи}}
    \fancyfoot[C]{ \noindent\rule{\textwidth}{0.4pt} \thepage }
	
	\newpage
	
	\pagenumbering{arabic}

\section*{Задача 1}
    
    \par \textsf{Найти длины волн (мкм), частоты (Гц) и энергии(эВ) для 7 цветов диапазона,}\\
    \par \textsf{видимого глазом человека излучения.}\\
    
    \par 
        \begin{enumerate}
            \item \textcolor[rgb]{1, 0, 0}{Красный}\\
            длина волны {$\lambda$}: 690 нм\\
            частота {$\nu$}: {$4,35 \cdot 10^{14}$} Гц\\
            энергия {$\hbar \omega$}: 1,8 эВ
            \item \textcolor[rgb]{1, 0.65, 0}{Оранжевый}\\
            длина волны {$\lambda$}: 610 нм\\
            частота {$\nu$}: {$5,00 \cdot 10^{14}$} Гц\\
            энергия {$\hbar \omega$}: 2,0 эВ
            \item \textcolor[rgb]{1, 1, 0}{Жёлтый}\\
            длина волны {$\lambda$}: 580 нм\\
            частота {$\nu$}: {$5,02 \cdot 10^{14}$} Гц\\
            энергия {$\hbar \omega$}: 2,1 эВ
            \item \textcolor[rgb]{0, 1, 0}{Зелёный}\\
            длина волны {$\lambda$}: 530 нм\\
            частота {$\nu$}: {$5,70 \cdot 10^{14}$} Гц\\
            энергия {$\hbar \omega$}: 2,3 эВ
            \item \textcolor[rgb]{0, 0.75, 1}{Голубой}\\
            длина волны {$\lambda$}: 490 нм\\
            частота {$\nu$}: {$6,12 \cdot 10^{14}$} Гц\\
            энергия {$\hbar \omega$}: 2,5 эВ
            \item \textcolor[rgb]{0, 0, 1}{Синий}\\
            длина волны {$\lambda$}: 460 нм\\
            частота {$\nu$}: {$6,52 \cdot 10^{14}$} Гц\\
            энергия {$\hbar \omega$}: 2,7 эВ
            \item \textcolor[rgb]{0.6, 0, 1}{Фиолетовый}\\
            длина волны {$\lambda$}: 420 нм\\
            частота {$\nu$}: {$7,10 \cdot 10^{14}$} Гц\\
            энергия {$\hbar \omega$}: 3,0 эВ
        \end{enumerate}

\vspace{0.8cm}

\section*{Задача 2}
    
    \par \textsf{Найти {$\lambda_m$} для АЧТ при t = $40^{\circ}$ C и $t = 6000^{\circ}$ C.}\\ 
    \par \textsf{Почему основной цвет растительности на Земле зеленый?}\\
    
    \par из закона Вина: {$\lambda_m = \frac{2900}{T}$}\\
    
    \par {$T_1 = 313$} K, {$T_2 = 6273$} K {$\longmapsto \lambda_{m_1} = 927$} нм, {$\lambda_{m_2} = 460$} нм\\
    
    \par растениям более эернетически выгодно отражать именно зелёный и сине-зелёный свет,\\
    \par посколько пик излучения Солнца как раз приблизительно 460-490 нм.

\vspace{1cm}

\section*{Задача 3}
    
    \par \textsf{Оценить суммарную мощность излучения (Вт), испускаемую Вами при нормальной температуре (t = $36,6^{\circ}$ C)}\\
    \par \textsf{и в состоянии болезни (t = $42^{\circ}$ C).}\\

    \par
    
\section*{Задача 4}
    
    \par \textsf{При классическом представлении Э-М поля, при какой его плоской поляризации}\\
    \par \textsf{(в плоскости падения или перпендикулярно ей) выход фотоэлектронов будет больше при всех равных}\\
    \par \textsf{других параметров излучения и фотокатода.}\\
    
    \par
    
\section*{Задача 5}
    
    \par \textsf{Определите красную длину волны фотоэффекта на алюминиевом фотокатоде, найдя его работу выхода. }\\
    \par \textsf{Найдите {$E_{MAX}$} фотоэлектрона, выбитого из алюминиевого фотокатода 4-ой гармоникой лазера на неодиме.}\\
    
    \par 

\section*{Задача 6}
    
    \par \textsf{Постройте линейную функцию запирающего потенциала от частоты}\\
    \par \textsf{падающего на фотокатод излучения {$U_{\hbox{зап}} = kV + b$}.}\\
    \par \textsf{Выразите k и b через константы и параметры фотокатода и покажите их на графике.}\\
    
    \par 

\section*{Задача 7}
    
    \par \textsf{Покажите, что поглощение / излучение свободного фотона свободным электроном - }\\
    \par \textsf{процесс, запрещенный законами сохранения.}\\
    
    \par 

\section*{Задача 8}
    
    \par \textsf{Определите изменение длины волны излучения при рассеянии его на пучке встречных релятивистских электронов,}\\
    \par \textsf{считая, что в результате неупругого столкновения с фотоном электрон часть своей кинетической энергии}\\
    \par \textsf{передал фотону, который отразился назад от релятивистского зеркала налетающих электронов.}\\
    \par \textit{Такой эффект получил название обратного комптон-эффекта. Именно обратным комптон-эффектом}\\
    \par \textit{удается, в частности, объяснить рентгеновское излучение космических объектов,}\\
    \par \textit{ и так на заре лазерной физики хотели получить рентгеновское лазерное излучение.}\\
    
    \par 

\section*{Задача 9}
    
    \par \textsf{Найдите и запишите выражения для вариационного принципа Ферма для оптики и вариационного принципа}\\
    \par \textsf{Мопертюи-Лагранжа для механики массовой частицы. Сравните их и попробуйте найти аналогии.}\\
    
    \par 

\section*{Задача 10}
    
    \par \textsf{Выпишите выражение для физической величины \textsc{действие} (S).}\\
    \par \textsf{Найдите ее размерность и сравните с размерностью постоянной Планка h.}\\
    \par \textsf{Запишите фазу плоской волны и фазу волновой функции через S/ h и сравните их временные}\\
    \par \textsf{и пространственные части.}\\
    
    \par 

\section*{Задача 11}
    
    \par \textsf{Как по картинке миража понять на юге или на севере это происходит?}\\
    
    \par 

\section*{Задача 12}
    
    \par \textsf{Оцените период кристаллической решетки никеля, если дифракционная картина типа Лауэ или Брега}\\
    \par \textsf{происходит с электронами, разогнанными разностью потенциалов в 150 эВ.}\\
    
    \par 

\section*{Задача 13}
    
    \par \textsf{Вычислить спектральное фурье - преобразование от функция временной когерентности}\\
    
    \par 

\section*{Задача 14}
    
    \par \textsf{Найти выражение для аксиальных мод пустого резонатора и константу из предыдущего равенства}\\
    \par \textsf{для пустого резонатора длины L.}\\
    
    \par 

\section*{Задача 15}
    
    \par \textsf{Оценить длину продольной когерентности излучения АЧТ в
длинах волн для комнатной температуры}\\
    \par \textsf{и для температуры короны Солнца.}\\
    
    \par 

\section*{Задача 16}
    
    \par \textsf{Оценить плотность мощности излучения, создаваемую
лазером на АИГ + Nd , имеющего диаметр}\\
    \par \textsf{выходной диафрагмы 2 мм и мощность импульса 100мДж на мишени на расстоянии 5 км.}\\
    
    \par 

\section*{Задача 17}
    
    \par \textsf{Найти температуру АЧТ, при которой параметр вырождения его излучения равен единице в видимом диапазоне.}\\
    
    \par 



\end{document}